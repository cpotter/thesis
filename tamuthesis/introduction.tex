%%%%%%%%%%%%%%%%%%%%%%%%%%%%%%%%%%%%%%%%%%%%%%%%%%%
%
%  New template code for TAMU Theses and Dissertations starting Fall 2012.
%  For more info about this template or the
%  TAMU LaTeX User's Group, see http://www.howdy.me/.
%
%  Author: Wendy Lynn Turner
%	 Version 1.0
%  Last updated 8/5/2012
%
%%%%%%%%%%%%%%%%%%%%%%%%%%%%%%%%%%%%%%%%%%%%%%%%%%%

%%%%%%%%%%%%%%%%%%%%%%%%%%%%%%%%%%%%%%%%%%%%%%%%%%%%%%%%%%%%%%%%%%%%%%
%%                           INTRODUCTION
%%%%%%%%%%%%%%%%%%%%%%%%%%%%%%%%%%%%%%%%%%%%%%%%%%%%%%%%%%%%%%%%%%%%%


\pagestyle{plain} % No headers, just page numbers
\pagenumbering{arabic} % Arabic numerals
\setcounter{page}{1}


\chapter{\uppercase {Introduction}}
Image synthesis is the process of generating images.  For the purpose of this thesis, Imaeg Synthesis will be considered as the process of generating images from virtual 3D scenes using a computer.  Computer graphics studios like Pixar and Dreamworks rely on propriety image synthesis pipelines to create photorealistic animations for their films.  The final product that is released to the public is a direct result of the image synthesis process.  Implementing a ray tracer is a daunting task but worth the effort for artists who are looking to make a career in professional computer-graphics lighting.  Performing the steps required to create images from a ray tracer is invaluable to understanding how to optimize render time while still creating high quality images.

Writing a rendering program can be intimidating for an artist. Rendering programs consist of two fundamental parts: theory and implementation.  The complex vector/matrix math theories needed to compute intersections within a 3D scene can be difficult to understand by themselves; coupled with proper programming language jargon can multiply the difficulty.  In addition to remembering linear algebra mathematical processes, artists need to be concerned about code syntax and segmentation faults (generally an attempt to access memory that the CPU cannot physically address), which can be demoralizing to even the most experienced computer scientist.  Another hurdle for artists is susceptibility to become distracted from the task at hand by focusing on the logistics of implementing code as opposed to gaining needed experience or experimenting with the mathematics of image synthesis.

Most students, when implementing a ray tracer, are siphoned into implementing their image synthesis programs in C++, perhaps because of tradition or familiarity from previous class work.  C++ can be a tricky language with many aspects of preparation needed before programming can begin.  C++ programs can be executed extremely quickly, which is needed for \textit{professional} image synthesis; the process of \textit{learning} ray tracing theory, however, does not necessarily benefit from a high-speed processing program.  This thesis has attempted to provide guidelines for introducing programming languages and techniques related to rendering so artists can spend less time on the implementation aspects of the process and more time on the theoretical experience of image synthesis.  In addition to the C++ language, Processing, Python, and RenderMan were evaluated for their potential as learning tools.

The qualitative results were generated by me, Christopher Potter.  My background in scripting languages was informal prior to my admittance into the Visualization Department at Texas A\&M.  I hold a BFA from the Rochester Institute of Technology with a minor in Computer Science.  My experience with computer languages, however, was not at a point where I felt comfortable writing simple scripts, nevermind an entire image synthesis program!  Firsthand results recorded in this thesis are notes and ``hiccups" encountered through each implementation step of the ray tracing process.  My goal is that these results will demostrate to student and teacher alike the complications that accompany each programming language and will help them overcome those obstacles in order to produce great art.
