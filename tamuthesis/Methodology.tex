%%%%%%%%%%%%%%%%%%%%%%%%%%%%%%%%%%%%%%%%%%%%%%%%%%%
%
%  New template code for TAMU Theses and Dissertations starting Fall 2012.
%  For more info about this template or the
%  TAMU LaTeX User's Group, see http://www.howdy.me/.
%
%  Author: Wendy Lynn Turner
%	 Version 1.0
%  Last updated 8/5/2012
%
%%%%%%%%%%%%%%%%%%%%%%%%%%%%%%%%%%%%%%%%%%%%%%%%%%%

%%%%%%%%%%%%%%%%%%%%%%%%%%%%%%%%%%%%%%%%%%%%%%%%%%%%%%%%%%%%%%%%%%%%%%%
%%%                           METHODOLOGY
%%%%%%%%%%%%%%%%%%%%%%%%%%%%%%%%%%%%%%%%%%%%%%%%%%%%%%%%%%%%%%%%%%%%%%

\chapter{\uppercase {Methodology}}
The methods I used were straightforward.  For each programming language notes were kept that outlined the difficulties and roadblocks caused specific to the language.  The same programming theories and fundamentals were utilized in order to provide continuity between each programming process.  Due to its helpfulness in managing and organizing large coding projects, the object-oriented approach to programming was taken.  This means that classes with specific variables and data-structures were defined so that instances of objects could be constructed at runtime.  Wherever possible, the image synthesis process was sub-categorized in order to better organize the information needed for the process. This led to shorter development time and also contributed to quicker troubleshooting after the code was written.

The program written for each language used a similar data structure and was modelled after the same overall design.  For each language a similar data structure was also strived for, shown in Figure \ref{fig:RayTracerDataStructure}.

A dynamic data structure that resizes as objects are added or removed along with an agnostic data type was desired for this project.  As we discussed next section, this approach provided a solution that mirrored the functionality of a ray caster.

In addition to having a program architecture and data structure, the milestones I established that segment image synthesis theory within four categories also provided a structured approach to analyzing and reporting the conclusions used for each language. Since each program was modeled after this conceptualization, measurements of success and difficulty were more clearly defined. More details on the milestones are discussed in the following section.

By determining the level of difficulty of implementing the program's data structures and class design, not to mention using that design to further implement the image synthesis milestones of Preliminary Preparations, Direct Illumination, Ray Tracing/Distributed Ray Tracing, and Indirect Illumination, a detailed report has been compiled that informs the results collected and reported in this thesis, all found in the next section.