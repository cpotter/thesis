\makeglossaries

\newglossaryentry{class}{

    name = Class,
    description = {the prototype for an object in an object-oriented language.  May also be considered to be a set of object which share a common structure and behaviour. The structure of a class is determined by the class variables which represent the state of an object of that class and the behavior is given by a set of methods associated with the class}
}

\newglossaryentry{array}{

    name = Array,
    description = {An array is similar data save on a computer system in sequential form}
    
}

\newglossaryentry{inheritance}{
    
    name = Inheritance,
    description = {when an object or class if based on another object or class, using the same implementation}

}

\newglossaryentry{superclass}{

    name = Superclass,
    description = {in the structure of inheritance, it is a parent class which establishes the base implementation of an object or class}
}

\newglossaryentry{subclass}{

    name = Subclass
    description = {in the structure of inheritance, it is the child class which inherits the base implementation of an object or class}
}

\newglossaryentry{polymorphism}{

    name = Polymorphism,
    description = {the provision of a single interface to entities of different types.  THe polymorphic type is atpe whose operations can also be applied to values of some other type or types.}

}

\newglossaryentry{pseudocode}{

    name = Pseudocode,
    description = {a notation resembling a simplified programming language, used in program design.}

}

\printglossary